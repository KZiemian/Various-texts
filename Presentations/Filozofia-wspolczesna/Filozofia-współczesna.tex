% Autor: Kamil Ziemian

% --------------------------------------------------------------------
% Podstawowe ustawienia Beamera i używane pakiety
% --------------------------------------------------------------------
\RequirePackage[l2tabu, orthodox]{nag} % Wykrywa przestarzałe i niewłaściwe
% sposoby używania LaTeXa. Więcej jest w l2tabu English version.

\documentclass{beamer}  % Klasa dokumentu
\mode<presentation>  % Rodzaj tworzonych slajdów Beamera
\usetheme{Warsaw}  % Temat graficzny

\setbeamertemplate{headline}{}  % Usuwa nagłówek
\setbeamersize{text margin left=3mm}  % Wielkość lewego marginesu
\setbeamersize{text margin right=3mm}  % Wielkość prawego marginesu
\setbeamertemplate{navigation symbols}{}  % Usuwa ikony nawigacji w prawym
% dolnym rogu



\usepackage[polish]{babel}  % Tłumaczy na polski teksty automatyczne LaTeXa
% i pomaga z typografią.
\usepackage[MeX]{polski}  % Polonizacja LaTeXa, bez niej będzie pracował
% w języku angielskim.
\usepackage[utf8]{inputenc}  % Włączenie kodowania UTF-8, co daje dostęp
% do polskich znaków.
\usepackage{lmodern}  % Wprowadza fonty Latin Modern.
\usepackage[T1]{fontenc}  % Potrzebne do używania fontów Latin Modern.



% ----------------------------
% Pakiety napisane przez użytkownika.
% Mają być w tym samym katalogu to ten plik .tex
% ----------------------------
\usepackage{latexshortcuts}



% ----------------------------
% Pakiet "hyperref"
% Polecano by umieszczać go na końcu preambuły.
% ----------------------------
\usepackage{hyperref}  % Pozwala tworzyć hiperlinki i zamienia odwołania
% do bibliografii na hiperlinki.





% --------------------------------------------------------------------
\title[Filozofia współczesna]{Filozofia współczesna}
\subtitle{Czyli jak to się jawi amatorowi}

\author{Kamil Ziemian \\
  \texttt{kziemianfvt@gmail.com}}

\institute{Uniwersytet Jagielloński w~Krakowie, \\
  Wydział Fizyki, Astronomii i~Informatyki Stosowanej}

% \date[28.05.2012]{28 maja 2012}
% --------------------------------------------------------------------





% ####################################################################
% Początek dokumentu
\begin{document}
% ####################################################################



% ######################################
\begin{frame}
  \titlepage
\end{frame}
% ######################################



% ######################################
\section[]{Wstęp i~ostrzeżenia}
% ######################################



% ##########
\begin{frame}
  \frametitle{Wstęp i~ostrzeżenia}

  \begin{block}{O co w tym chodzi?}
    Poopowiadać trochę o~filozofii współczesnej, aby~pomóc tym
    co~chcących ją zgłębić i~poszukujących jakiegokolwiek punktu
    zaczepienia. Albo po~prostu chce o~niej posłuchać. \\
    Co nie znaczy, że~to seminarium jest dobrym wyborem.
  \end{block}

  \begin{block}{Ostrzeżenia i~uwagi}
    \begin{itemize}
    \item[--] W~tej opowieści będzie wiele luk, zapewne też wiele
      błędów.
    \item[--] Filozofia z~okresu po~roku 2000 będzie w~ilościach
      śladowych.
    \item[--] Proszę również o~wybaczenie literówek, niepoprawnej
      składni,~etc. Postaram~się w~przyszłości je poprawić.
    \item[--] Proszę o~zgłoszenie wszystkich uwag, bowiem wierzę, że
      dobrze jest uczyć się na błędach.
    \end{itemize}
  \end{block}

\end{frame}
% ##########



% ##########
\begin{frame}
  \frametitle{Zarys planu}

  \begin{block}{}
    \begin{enumerate}
    \item Pierwszy filozof od~którego trzeba zacząć i~jego myśl.
    \item Jego naśladowcy i~ich krytycy.
    \item Rozłam dwudziestowieczny.
    \item Dwa giganci XX~wiecznej filozofii.
    \item Tłum pominiętych myślicieli.
    \end{enumerate}
  \end{block}

\end{frame}
% ##########





% ######################################
\section[]{Zacznijmy prawie od~początku\ldots}
% ######################################


% ##########
\begin{frame}
  \frametitle{Zacznijmy prawie od początku\ldots}

  \begin{block}{Złota zasada}
    \begin{figure}
      \centering

      \includegraphics[height=1.6in, width=1.3in]{Plato_Aristotle.jpg}

      \uncover<2->{\caption{Platon (427--347 BC), Arystoteles
          (384--322 BC)}}
    \end{figure}
  \end{block}

  \begin{block}{}
    O~czymkolwiek~się mówi w~filozofii, zawsze można zacząć od~tych
    dwóch panów i~dojść logicznie do wybranego tematu.
  \end{block}

\end{frame}
% ##########



% ##########
\begin{frame}
  \frametitle{Aż tyle czasu nie mamy}

  \begin{block}{Filozof od którego trzeba zacząć}
    \begin{figure}
      \centering

      \includegraphics[height=1.6in, width=1.4in]{Immanuel_Kant_1.jpg}
      \caption{Immanuel Kant (1724--1804)}
    \end{figure}
  \end{block}

  \begin{block}{W~mojej opinii}
    \pause Najbardziej wpływowy filozof dwóch ostatnich wieków.
  \end{block}

\end{frame}
% ##########



% ##########
\begin{frame}
  \frametitle{Co warto wiedzieć o~Kancie\ldots}

  \begin{block}{O~Luter, Kartezjuszu i~Rousseau.}
    \textbf{Są~oni ojcami tego co Gabriel S\'{e}ailles nazwał
      współczesnym sumieniem. Pomijam Immanuela Kanta, który stoi
      na~skrzyżowaniu prądów duchowych, płynących od~tych trzech
      ludzi. Filozof z~Królewca stworzył niejako scholastyczną
      armaturę współczesnej myśli.} --~Jacques Maritain ,,Trzej
    reformatorzy''.
  \end{block}

  \begin{block}{Życiorys Kanta}
    \begin{itemize}
    \item[--] Urodził się, żył i umarł w mieście K\"{o}nisberg
      (Królewiec) w~Prusach.
    \item[--] Był profesorem na~tamtejszym uniwersytecie.
    \item[--] Tym samy jeden z~dwóch najważniejszych filozofów którzy
      pomiędzy~XV, a~XX wiekiem parali~się tym zajęciem.
    \item[--] Nigdy się nie ożenił.
    \item[--] Legendarny pedant.
    \item[--] Domator pierwszej kategorii.
    \item[--] Główne prace: \\
      Krytyka czystego rozumu 1781, \\
      Krytyka praktycznego rozumu 1788, \\
      Krytyka władzy sądzenia 1791.
    \end{itemize}
  \end{block}

\end{frame}
% ##########



% ##########
\begin{frame}
  \frametitle{\ldots i jego filozofii?}

  \begin{block}{Główne źródła myśli Kanta}
    \begin{itemize}
    \item[--] Pietystyczne wychowanie.
    \item[--] Akademicka filozofia niemiecka.
    \item[--] Polemika z Humem.
    \item[--] Potrzeba pogodzenia nauki i~wiary. \\
      Chodzi tu raczej o~fizykę Newtona niż biologię, to~jest osobna
      historia.
    \item[--] Filozofia oświecenia. Warty jest podkreślenia wpływ
      Rousseau.
    \item[--] Prawdopodobnie polemika z~filozofią scholastyczną (???).
    \item[--] Emanuel Swedenborg?
    \end{itemize}
  \end{block}

  \begin{block}{Ważne}
    Można powiedzieć, że \emph{Krytyka czystego rozumu} jest wykładem
    metafizyki, \emph{Krytyka praktycznego rozumu} etyki,
    zaś~\emph{Krytyka władzy sądzenia} --~estetyki. Wszystkie te nurty
    myśli Kanta są ważne, jednak tylko pierwszą będę w~stanie naprawdę
    poruszyć.
  \end{block}
\end{frame}
% ##########



% ##########
\begin{frame}
  \frametitle{\ldots i jego filozofii?}

  \begin{block}{Crashcourse filozofii Kanta}
    \begin{itemize}
      \pause
    \item[--] Po pierwsze i najważniejsze: Kant był idealistą!
    \item[--] Ale co to znaczy?
    \item[--] Kantowska kategorie sądów: \\
      a) analityczne, syntetyczne; \\
      b) a priori (łac. z~tego, co~wcześniej), a posteriori
      (łac.~z~tego, co~potem).
    \item[--] Czas i przestrzeń, arytmetyka i~geometria.
    \item[--] Ding an sich. Wielki niszczyciel.
    \item[--] Koncepcja moralna: uniwersalizm.
    \end{itemize}
  \end{block}
\end{frame}
% ##########



% ##########
\begin{frame}
  \frametitle{Niemiecki Idealizm}

  \begin{block}{Pierwsze uderzenie Kanta}
    Idealizm Kanta dał, między innymi, początek ruchowi jaki nosi
    nazwę Niemiecki Idealizm. Oprócz samej myśli Kanta oddziaływały tu
    niewątpliwie jego źródła, takie jak pietyzm.
  \end{block}
  \pause

  \begin{block}{Warto się zastanowić}
    Kant był pierwszym wielkim filozofem piszącym po niemiecku (bardzo
    nieczytelnym niemieckim), w czasach gdy zaczęła się rodzić
    współczesna koncepcja niemieckości.
  \end{block}

\end{frame}
% ##########



% ##########
\begin{frame}
  \frametitle{Niemiecki Idealizm}

  \begin{block}{Pierwszy etap}
    \begin{figure}
      \centering

      \includegraphics[height=1.6in, width=1.5in]{Fichte.jpg}
      \includegraphics[height=1.6in, width=1.4in]{Schelling.jpg}
      \caption{Johann~Gottlieb~Fichte (1762--1814), Friedrich
        Schelling (1775--1854)}
    \end{figure}
  \end{block}

\end{frame}
% ##########



% ##########
\begin{frame}
  \frametitle{Niemiecki Idealizm}

  \begin{block}{Kulminacja}
    \begin{figure}
      \centering

      \includegraphics[height=1.6in, width=1.4in]{Hegel.jpg}

      \caption{Georg Wilhelm Friedrich Hegel (1770--1831)}
    \end{figure}
  \end{block}

  \begin{block}{}
    Prawdopodobnie drugi po Kancie najbardziej wpływowy filozof z~tu
    przedstawionych.
  \end{block}

\end{frame}
% ##########



% ##########
\begin{frame}
  \frametitle{Niemiecki Idealizm}

  \begin{block}{Idee i~wpływ}
    \begin{itemize}
    \item[--] Wpływ na wiek XIX był po~prostuj ogromny.
    \item[--] Relacja człowiek-świat.
    \item[--] Koncepcja filozofii dziejów, w zupełnie nowym ujęciu.
    \item[--] Miejsce jednostki w świecie i państwie.
    \item[--] Osiągnięcie absolutu.
    \item[--] Romantyzm.
    \item[--] Związek z~gnostycyzmem.
    \end{itemize}
  \end{block}

  \begin{block}{A co na to Kant?}
    \pause Dwie rzeczy napełniają umysł coraz to nowym i wzmagającym
    się podziwem i czcią, im częściej i trwalej się nad nimi
    zastanawiamy: niebo gwiaździste nade mną i~prawo moralne we mnie.
  \end{block}

  \begin{block}{Fenomenologia Ducha}
    Wydane w~1807 roku dzieło Hegla \textbf{Die Ph\"{a}nomenologie \newline
    des Geistes} to jedno z~najważniejszych dzieł filozoficznych
    ostatnich stuleci. Według Hegla fenomenologia to \emph{nauka
    o~doświadczaniu świadomości}.
  \end{block}
  \pause

  \begin{block}{Kilka ważny elementów}
    \begin{itemize}
    \item ,,Des Geistes''.
      % \item Książka porusza prawie każdy temat.
    \item Filozofia dialektyczna.
    \item Samoświadomość.
    \item Dialektyka pana i~niewolnika.
    \end{itemize}
  \end{block}

  \begin{block}{Phainomenon}
    Z greki obserwowany, zjawisko. Istniej również kierunek filozofii
    zwany fenomenologią (o~tym potem).
  \end{block}
  \pause

  \begin{block}{Wpływ tego nurtu ciężko oszacować i~przecenić}
    \begin{itemize}
    \item Jest jednym z~najważniejszych źródeł romantyzmu
      europejskiego. Niewątpliwy związek z nim mieli Samuel Taylor
      Coleridge, Thomas De Quincey, Juliusz Słowacki, jednak na nich
      lista się na pewno nie kończy.
    \item Filozofia historii.
    \item Rekonstrukcja gnostycyzmu przez Schellinga.
    \item Neoheglizm był żywy co najmniej do końca XIX wieku.
    \end{itemize}
  \end{block}

\end{frame}
% ##########



% ##########
\begin{frame}
  \frametitle{Ale, natura ludzka jest taka\ldots}
  \pause

  \begin{block}{Antyheglizm}
    \begin{figure}
      \centering

      \includegraphics[height=1.6in,
      width=1.4in]{Soren_Kierkegaard_1.jpg}
      \includegraphics[height=1.6in, width=1.2in]{Karl_Marx_1.jpg}
      \pause

      \caption{Søren Kierkegaard (1813--1855), Karl Marx (1818--1883)}
    \end{figure}
  \end{block}

\end{frame}
% ##########



% ##########
\begin{frame}
  \frametitle{Ich bardziej znane wizerunki}

  \begin{block}{Antyheglizm}
    \begin{figure}
      \centering

      \includegraphics[height=1.6in,
      width=1.2in]{Soren_Kierkegaard_2.jpg}
      \includegraphics[height=1.6in, width=1.2in]{Karl_Marx_2.jpg}
      \caption{Søren Kierkegaard (1813--1855), Karl Marx (1818--1883)}
    \end{figure}
  \end{block}

\end{frame}
% ##########



% ##########
\begin{frame}
  \frametitle{Antyheglizm kantowski Schopenhauer i~jego następstwa}

  \begin{block}{}
    \begin{figure}
      \centering

      \includegraphics[height=1.6in,
      width=1.4in]{Arthur_Schopenhauer_1.jpg}
      \includegraphics[height=1.6in,
      width=1.2in]{Friedrich_Nietzsche_1.jpg} \pause

      \caption{Arthur Schopenhauer (1788--1860), Friedrich Nietzsche
        (1844--1900)}
    \end{figure}
  \end{block}

\end{frame}
% ##########



% ##########
\begin{frame}
  \frametitle{I~ich bardziej znane wizerunki}

  \begin{block}{}
    \begin{figure}
      \centering

      \includegraphics[height=1.6in,
      width=1.4in]{Arthur_Schopenhauer_2.jpg}
      \includegraphics[height=1.6in,
      width=1.4in]{Friedrich_Nietzsche_2.jpg}

      \caption{Arthur Schopenhauer (1788--1860), Friedrich Nietzsche
        (1844--1900)}
    \end{figure}
  \end{block}

\end{frame}
% ##########



% ##########
\begin{frame}
  \frametitle{A że wpływowi Kanta nie ma końca}

  \begin{block}{Ameryka i\ldots}
    \begin{figure}
      \centering

      \includegraphics[height=1.6in, width=1.1in]{Ralph_Emerson.jpg}

      \caption{Ralph Waldo Emerson (1803--1882)}
    \end{figure}
  \end{block}

  \begin{block}{}
    Twórca pierwszej amerykańskiej filozofii zwanej
    \emph{transcendetalizmem}. Jej znajomość jest absolutnie niezbędna
    przy studiowaniu kultury Stanów Zjednoczonych.
  \end{block}

\end{frame}
% ##########



% ##########
\begin{frame}
  \frametitle{Zanim przejdziemy do zakończenia, mała odskocznia}

  \begin{block}{Kto jeszcze?}
    \begin{figure}
      \centering

      \includegraphics[height=1.6in,
      width=1.9in]{Ludwig_von_Mises.jpg}
      \includegraphics[height=1.6in,
      width=1.2in]{Hans_Hermann_Hoppe.jpg}
      \includegraphics[height=1.6in,
      width=1.2in]{Bronislaw_Malinowski.jpg}

      \caption{Ludwig von Mises (1881--1973), Hans-Hermann Hoppe
        \newline (1949--), Bronisław Malinowski (1884--1942)}
    \end{figure}
  \end{block}

\end{frame}
% ##########



% ##########
\begin{frame}
  \frametitle{Krótki przegląd pozostałych filozofów XIX w.}

  \begin{block}{Utylitaryzm i pozytywizm}
    \begin{figure}
      \centering

      \includegraphics[height=1.6in, width=1.2in]{Jeremy_Bentham.jpg}
      \includegraphics[height=1.6in,
      width=1.2in]{John_Stuart_Mill.jpg}
      \includegraphics[height=1.6in, width=1.2in]{August_Comte.jpg}

      \caption{Jeremy Bentham (1748--1832), John Stuart Mill
        (1806--1873), Auguste Comte (1798--1857)}
    \end{figure}
  \end{block}

\end{frame}
% ##########



% ##########
\begin{frame}
  \frametitle{Krótki przegląd pozostałych filozofów XIX w.}

  \begin{block}{Ameryka: transcendentalizm i pragmatyzm}
    \begin{figure}
      \centering

      \includegraphics[height=1.6in,
      width=1.2in]{Henry_David_Thoreau.jpg}
      \includegraphics[height=1.6in, width=1.2in]{CP}
      \includegraphics[height=1.6in, width=1.2in]{William_James.jpg}
      
      \caption{Henry David Thoreau (1748--1832), Charles Sanders
        Peirce (1839--1914), William James (1842--1910)}
    \end{figure}
  \end{block}

\end{frame}
% ##########



% ##########
\begin{frame}
  \frametitle{Krótki przegląd pozostałych filozofów XIX~w.}

  \begin{block}{Wielki nauczycie}
    \begin{figure}
      \centering

      \includegraphics[height=1.6in, width=1.2in]{Franz_Brentano.jpg}

      \caption{Franz Brentano (1838--1917)}
    \end{figure}
  \end{block}

\end{frame}
% ##########



% ##########
\begin{frame}{Na zakończenie}
  \frametitle{Krótki przegląd pozostałych filozofów XIX w.}

  \begin{block}{Zapomniany Wiktoriański gigant}
    \begin{figure}
      \centering

      \includegraphics[height=1.6in, width=1.1in]{Herbert_Spencer.jpg}

      \caption{Herbert Spencer (1820--1903)}
    \end{figure}
  \end{block}

\end{frame}
% ##########



% ##########
\begin{frame}
  \frametitle{Emanuel Swedenborg}
  \begin{block}{}
    \begin{figure}
      \centering

      \includegraphics[height=1.6in,
      width=1.3in]{Emanuel_Swedenborg.png}

      \caption{Emanuel Swedenborg (1688--1772)}
    \end{figure}
  \end{block}

  \begin{block}{Kto to jest?}
    Najwybitniejszy mag czasów nowożytnych. I~Oświecenia, a~to duże
    osiągnięcie.
  \end{block}

\end{frame}
% ##########



% ##########
\begin{frame}

  \begin{block}{Wpływy i opinie}
    \begin{itemize}
    \item[--] Immanuel Kant \emph{Tr\"{a}ume eines Geistersehers}
      (\emph{Sny o~duchu widzącym}) 1766~r.
    \item[--] Ralph Waldo Emerson \emph{Swedenborg;~or, the~Mystic}.
    \item[--] Daisetsu Teitaro Suzuki (1870--1966) \emph{Swedenborg
        --~Budda Północy}.
    \item[--] Czesław Miłosz. Wiersz \emph{Piekła Swedenborga}, esej
      w~\emph{Ogrodzie nauk}, wywiad we wstępie do książki S.~Toksvig
      \emph{Emanuel Swedenborg --~uczony i~mistyk}. \textbf{Nawet
        w~jego opisach piekła widać oko inżyniera}.
    \item[--] Joseph Campbell, \emph{Bohater o tysiącu twarzy},
      1949~r. \\ (Ale to temat na inne seminarium.)
    \end{itemize}
  \end{block}

\end{frame}
% ##########



% ##########
\begin{frame}
  \frametitle{Ale przecież\ldots}
  \begin{block}{Romantyk wygląda tak}
    \begin{figure}
      \centering

      \includegraphics[height=1.3in,
      width=1.1in]{Caspar_Friedrich_Wanderer.jpg} \pause
      \includegraphics[height=1.3in, width=1.0in]{Immanuel_Kant_2.jpg}
    \end{figure}
    % A nie tak.
  \end{block}
  \pause

  \begin{block}{Oddajmy głos Kantowi}
    \textbf{Dwie rzeczy napełniają umysł coraz to nowym i~wzmagającym
      się podziwem i czcią, im częściej i trwalej się nad nimi
      zastanawiamy: niebo gwiaździste nade mną i prawo moralne we
      mnie.} --~\emph{Krytyka praktycznego rozumu} (kto czytał
    \emph{Linię oporu} Dukaja?).
  \end{block}

\end{frame}
% ##########



% ##########
\begin{frame}
  \frametitle{Zanim przejdziemy do sedna, mała odskocznia}

  \begin{block}{Czy Freud był kantystą?}
    \begin{figure}
      \centering

      \includegraphics[height=1.6in, width=1.2in]{Sigmund_Freud.jpg}

      \caption{Sigmund Freud (1856--1939)}
    \end{figure}
  \end{block}

\end{frame}
% ##########



% ##########
\begin{frame}
  \frametitle{Pole walki ustawione, możemy przejść do~dzieła}

  \begin{block}{Wielki dwudziestowieczny rozłam}
    \pause
    \begin{itemize}
    \item Kant zdefiniował XIX stulecie w stopniu niebywałym.
    \item Dlatego też, prawem ludzkiej natury, filozofia
      dwudziestowieczna została w pierwszym rzucie zdefiniowana przez
      próby odrzucenia modelu filozofii Kanta.
    \item Udało się to w sposób różny i nie sięgnęło wszystkich sfer
      na których odcisnął swe piętno wielki Immanuel.
    \item Zrodziło to dwie wielkie szkoły filozofii, które zdominowały
      wiek XX.
    \item Obu pierwsze przebłyski pojawiły się w Niemczech.
    \item Znane są jako filozofia analityczna i kontynentalna.
    \item Jest to prawdopodobnie największy i najbardziej skandaliczny
      rozłam w~historii filozofii.
    \end{itemize}
  \end{block}

\end{frame}
% ##########



% ##########
\begin{frame}
  \frametitle{Osoby dramatu}
  \pause

  \begin{block}{Pierwsze fala}
    \begin{figure}
      \centering

      \includegraphics[height=1.6in, width=1.4in]{Gottlob_Frege.jpg}
      \includegraphics[height=1.6in, width=1.4in]{Edmund_Husserl.jpg}

      \caption{Gottlob Frege (1848--1925), Edmund Husserl
        (1859--1938)}
    \end{figure}
  \end{block}

\end{frame}
% ##########



% ##########
\begin{frame}
  \frametitle{Osoby dramatu}

  \begin{block}{Pierwsze fala}
    \begin{figure}
      \centering

      \includegraphics[height=1.6in, width=1.2in]{GEMoore.jpeg}
      \includegraphics[height=1.6in,
      width=1.4in]{Bertrand_Russell.jpeg}

      \caption{G.~E.~Moore (1873--1958), Bertrand Russell
        (1872--1970)}
    \end{figure}
  \end{block}

\end{frame}
% ##########



% ##########
\begin{frame}
  \frametitle{Spojrzenie perspektywiczne}

  \begin{block}{Podobieństwa i różnice}
    \begin{itemize}
    \item Obie zaczęły się w Niemczech.
    \item Ich punktem wyjścia była logika.
    \item Jednak szkoła kontynentalna znalazła szybko inne tematy
      zainteresowań.
    \item Filozofia analityczna zdominowała do lat 60\dywiz tych świat
      anglosaski, kontynentalna -- cała resztę.
    \item W obu język zaczął odgrywać pierwszoplanową rolę.
    \item Plany ojców, w obu przypadkach, zostały wykończone przez
      najwybitniejszych synów.
    \item Jeżeli słyszał ktoś coś głupiego co robiona na
      średniowiecznych uniwersytetach, filozofowie ci co najmniej im
      dorównali.
    \end{itemize}
  \end{block}

\end{frame}
% ##########



% ##########
\begin{frame}{Dwaj giganci --~niszczyciele}
  \pause

  \begin{block}{Czyli\ldots}
    \begin{figure}
      \centering

      \includegraphics[height=1.6in,
      width=1.2in]{Martin_Heidegger.jpg}
      \includegraphics[height=1.6in,
      width=1.4in]{Ludwig_Wittgenstein.jpg} \pause

      \caption{Martin Heidegger (1889--1976), Ludwig Wittgenstein
        (1889--1951)}
    \end{figure}
  \end{block}

\end{frame}
% ##########



% ##########
\begin{frame}{Martin Heidegger}

  \begin{block}{Życiorys esencjonalny}
    \begin{itemize}
    \item Urodził się.
    \item Był nazistą.
    \item Umarł.
    \end{itemize}
  \end{block}

  \begin{block}{Życiorys filozoficzny}
    \begin{itemize}
      \pause
    \item Student Husserla.
    \item Ideowy Niemiec.
    \item Dokonał ,,zwrotu'' w swej filozofii.
      % \item Ustanowił dwudziestowieczną hermeneutykę. \pause
    \item Miał obsesje na punkcie etymologii i~pisał straszliwie
      niezrozumiale.
    \item Główne dzieło: \\
      \textbf{Seit und Zeit} (Bycie i~czas), 1927.
    \end{itemize}
  \end{block}

\end{frame}
% ##########



% ##########
\begin{frame}{Seit und Zeit}

  \begin{block}{Co warto wiedzieć}
    \begin{itemize}
    \item Jedna z~dwóch najważniejszych książek filozoficznych po roku
      1925.
    \item Światowy bestseller. M.in sześć tłumaczeń na język japoński.
    \item Zadedykowane Husserlowi.
    \item Dzieło fundacyjne współczesnego egzystencjalizmu.
    \item Jest to w zasadzie pierwsza część dwutomowego dzieła.
    \item Tom~II nigdy nie został napisany.
    \end{itemize}
  \end{block}
  \pause

  \begin{block}{Ciekawostka}
    \pause Światowa twarz egzystencjalizmu Jean-Paul Sartre (jak ktoś
    nie wierzy to~niech obejrzy ostatni odcinek \emph{Black Lagoon})
    wydał w~1943~r. książkę \emph{L'étre et le néant} (pl. \emph{Bycie
      i~nicość}).
  \end{block}

\end{frame}
% ##########



% ##########
\begin{frame}{Seit und Zeit}

  \begin{block}{Pytanie o sens bycia}
    \textbf{Czy w dzisiejszych czasach dysponujemy odpowiedzią na
      pytanie, co właściwie rozumiemy pod słowem ,,bytujący''? W
      żadnym razie nie. Dlatego trzeba na nowo postawić \emph{pytanie
        o sens bycia}. Czy jednak dzisiaj fakt, że nie rozumiemy
      wyrażenia ,,bycie'' jest naszym jedynym kłopotem? Bynajmniej.
      Dlatego też przede wszystkim musimy na nowo obudzić zrozumienie
      dla sensu tego pytania. Zamiarem niniejszej rozprawy jest
      konkretne opracowanie pytania o sens ,,bycia''. Prowizorycznym
      celem zaś będzie interpretacja \emph{czasu} jako możliwego
      horyzontu wszelkiego w ogóle rozumienia bycia.}
  \end{block}

  \begin{block}{Głębokie pytanie\ldots}
    O~co mu chodziło?
  \end{block}

\end{frame}
% ##########



% ##########
\begin{frame}{Seit und Zeit}

  \begin{block}{Uwagi techniczne}
    \begin{itemize}
    \item Heidegger używał silnie neologizmów, lub zwykłych słów
      niemieckich w zupełnie innym sensie.
    \item Przykład: Dasein.
    \item Jego styl jest trudny/nieludzko
      zagmatwany/niejasny/zagadkowy/bezsensowny/maskujący brak treści
      (niepotrzebne skreślić). Jako przykład podaję fragment z~jego
      dzieła \textbf{Zeit und~Seit} (\textbf{Czas i~bycie},
      ale~wymyślił), wybrany przez Jacques Derrida, będącego pod jego
      wielkim wpływem, jako motto jednej
      z~jego prac: \\
      \textbf{Wszechobecność wyistaczania ukazuje nam się najwyraźniej
        wówczas, gdy zauważymy, że również i właśnie odistaczanie
        określone jest poprzez owo, tajemnicze niekiedy wyistaczanie}.
    \end{itemize}
    \pause
  \end{block}

  \begin{block}{Teraz wyzwanie}
    O czym mówi \textbf{Seit und~Zeit}?
  \end{block}

\end{frame}
% ##########



% ##########
\begin{frame}{Seit und~Zeit}

  \begin{block}{Pytania}
    \begin{itemize}
      \pause
    \item Dlaczego nie ma drugiej części? \pause
    \item Jaki był jej wpływ?
    \end{itemize}
  \end{block}

\end{frame}
% ##########



% ##########
\begin{frame}{Ludwig Wittgenstein}

  \begin{block}{Życiorys}
    \begin{itemize}
    \item Urodził się w Wiedniu, w Cesarstwie Austro-Węgierskim.
    \item Jego rodzina ze strony miała żydowskie pochodzenie, ze
      strony matki czesko-słoweńskie
    \item Był jednym z dziewięciorga dzieci potentata stalowego
      i~jednego z~najbogatszych ludzi w~Europie.
    \item Początkowo pragnął zostać inżynierem aeronautyki.
    \item Nauka potrzebnej mu matematyki doprowadziło go do pytań o
      jej podstawy.
    \end{itemize}
  \end{block}

\end{frame}
% ##########



% ##########
\begin{frame}
  \begin{block}{Życiorys filozoficzny}
    \begin{itemize}
      \pause
    \item Jako jedyny chyba człowiek w pierwszej połowie XX w. czytał
      i brał na serio Schopenhauera.
    \item Problemy z podstawami matematyki prowadzą go do prac
      Frege`ego i Russela. Poznaje i zaprzyjaźnia się z oboma. \pause
    \item Studiuje pod Russelem. \pause
    \item 1914 wybucha I Wojna Światowa, Wittgenstein zaciąga~się do
      wojska Austro-Węgierski`ego. \pause
    \item W przerwach działań wojennych tworzy
      \textbf{Tractatus\ldots} i~wiele notatek filozoficznych, zostaje
      też dwukrotnie wyznaczony do~medalu za~odwagę.
    \item
    \end{itemize}
  \end{block}

\end{frame}
% ##########



% ##########
\begin{frame}{Na zakończenie}

  \begin{block}{Ci którzy zostali pominięci}
    \begin{figure}
      \centering

      \includegraphics[height=1.6in,
      width=1.2in]{John_Stuart_Mill.jpg}
      \includegraphics[height=1.6in, width=1.2in]{August_Comte.jpg}
      \includegraphics[height=1.6in, width=1.2in]{Charles_Pierce.jpg}

      \caption{John Stuart Mill (1806--1873), Auguste Comte
        (1798--1857), Charles Sanders Peirce (1839--1914)}
    \end{figure}
  \end{block}

\end{frame}
% ##########



% ##########
\begin{frame}{Na zakończenie}

  \begin{block}{Ci którzy zostali pominięci}
    \begin{figure}
      \centering

      \includegraphics[height=1.6in, width=1.2in]{William_James.jpg}
      \includegraphics[height=1.6in, width=1.2in]{Henri_Bergson.jpg}
      \includegraphics[height=1.6in,
      width=1.2in]{Alfred_North_Whitehead.jpg}

      \caption{William James (1842--1910), Henri-Louis Bergson
        (1859--1941), Alfred North Whitehead (1861--1947)}
    \end{figure}
  \end{block}

\end{frame}
% ##########



% ##########
\begin{frame}{Na zakończenie}

  \begin{block}{Ci którzy zostali pominięci}
    \begin{figure}
      \centering

      \includegraphics[height=1.6in,
      width=1.2in]{Maurice_Merleau_Ponty.jpg}
      \includegraphics[height=1.6in, width=1.3in]{John_Dewey.jpg}
      \includegraphics[height=1.6in,
      width=1.5in]{Willard_Van_Orman_Quine.jpg}

      \caption{Maurice Merleau-Ponty (1908--1961), John Dewey
        (1859--1961), Willard Van Orman Quine (1908--2000)}
    \end{figure}
  \end{block}

\end{frame}
% ##########



% ##########
\begin{frame}{Na zakończenie}

  \begin{block}{Ci którzy zostali pominięci}
    \begin{figure}
      \centering

      \includegraphics[height=1.6in, width=1.2in]{Rudolf_Carnap.jpg}
      \includegraphics[height=1.6in,
      width=1.1in]{Elizabeth_Anscombe.jpg}
      \includegraphics[height=1.6in, width=1.2in]{Karl_Poper.jpg}

      \caption{Rudolf Carnap (1891--1970), Gertrude Elizabeth Margaret
        Anscombe (1919--2001), Karl Popper (1902--1994)}
    \end{figure}
  \end{block}

\end{frame}
% ##########



% ##########
\begin{frame}{Na zakończenie}

  \begin{block}{Ci którzy zostali pominięci}
    \begin{figure}
      \centering

      \includegraphics[height=1.6in, width=1.3in]{Ayn_Rand.jpg}
      \includegraphics[height=1.6in, width=1.5in]{John_Rawls.jpg}
      \includegraphics[height=1.6in, width=1.1in]{Robert_Nozik.jpg}
      \caption{Ayn Rand (1905--1982), John Rawls (1921--2002), Robert
        Nozick (1938--2002)}
    \end{figure}
  \end{block}

\end{frame}
% ##########



% ##########
\begin{frame}{Na zakończenie}

  \begin{block}{Ci którzy zostali pominięci}
    \begin{figure}
      \centering

      \includegraphics[height=1.6in,
      width=1.2in]{Hans_Georg_Gadamer.jpg}
      \includegraphics[height=1.6in,
      width=1.3in]{Claude_Levi_Strauss.jpg}
      \includegraphics[height=1.6in,
      width=1.2in]{Alisdair_MacIntyre.jpg}
      
      \caption{Hans-Georg Gadamer (1900--2002), Claude Lévi-Strauss
        (1908--2009), Alasdair MacIntyre (1931--2007)}
    \end{figure}
  \end{block}

\end{frame}
% ##########



% ##########
\begin{frame}{I na ostatek}

  \begin{block}{Najważniejszy filozof drugiej połowy XX w.}
    \pause
    \begin{figure}
      \centering

      \includegraphics[height=1.6in,
      width=1.2in]{Jacques_Derrida_1.jpg}
      \includegraphics[height=1.6in,
      width=1.8in]{Jacques_Derrida_3.jpg}
      \includegraphics[height=1.6in,
      width=1.2in]{Jacques_Derrida_2.jpg}

      \caption{Jacques Derrida (1930--2004)}
    \end{figure}
  \end{block}

\end{frame}
% ##########





% ####################################################################
% ####################################################################
% Bibliografia
\bibliographystyle{alpha} \bibliography{Bibliography}{}


% ############################

% Koniec dokumentu
\end{document}