% Autor: Kamil Ziemian

% --------------------------------------------------------------------
% Podstawowe ustawienia Beamera i używane pakiety
% --------------------------------------------------------------------
\RequirePackage[l2tabu, orthodox]{nag} % Wykrywa przestarzałe i niewłaściwe
% sposoby używania LaTeXa. Więcej jest w l2tabu English version.

\documentclass{beamer}  % Klasa dokumentu
\mode<presentation>  % Rodzaj tworzonych slajdów Beamera
\usetheme{Warsaw}  % Temat graficzny

\setbeamertemplate{headline}{}  % Usuwa nagłówek
\setbeamersize{text margin left=3mm}  % Wielkość lewego marginesu
\setbeamersize{text margin right=3mm}  % Wielkość prawego marginesu
\setbeamertemplate{navigation symbols}{}  % Usuwa ikony nawigacji w prawym
% dolnym rogu



\usepackage[utf8]{inputenc}  % Włączenie kodowania UTF-8, co daje dostęp
% do polskich znaków.
\usepackage{lmodern}  % Wprowadza fonty Latin Modern.
\usepackage[T1]{fontenc}  % Potrzebne do używania fontów Latin Modern.



% ----------------------------
% Pakiety napisane przez użytkownika.
% Mają być w tym samym katalogu to ten plik .tex
% ----------------------------
\usepackage{latexshortcuts}
\usepackage{mathshortcuts}



% --------------------------------------
% Komendy do prostszego pisania prezentacji
% --------------------------------------
% --------------------------------------



% ----------------------------
% Pakiet "hyperref"
% Polecano by umieszczać go na końcu preambuły.
% ----------------------------
\usepackage{hyperref}  % Pozwala tworzyć hiperlinki i zamienia odwołania
% do bibliografii na hiperlinki.





% --------------------------------------------------------------------
\title[]{Julia}
\subtitle{2010s proposition for scientific (and other) programming}

\author{Kamil Ziemian \\
  \texttt{kziemianfvt@gmail.com} }

% \institute{Jagiellonian University in~Cracow}

\date[23 November 2018]{Seminar~of Field Theory Department \\
  23 November 2018}
% --------------------------------------------------------------------





% ####################################################################
% Początek dokumentu
\begin{document}
% ####################################################################



% ######################################
\begin{frame}
  \titlepage  % Tytuł całego tekstu
\end{frame}
% ######################################



% % ######################################
% \begin{frame}
%   \tableofcontents  % Spis treści
% \end{frame}
% % ######################################



% ######################################
\section[]{Why not learning Julia?}
% ######################################





% ##########
\begin{frame}
  \frametitle{Background information}

  \begin{block}{Basics}
    \begin{itemize}
    \item Julia has started as project in 2009, first release 0.1 in
      2012, version 1.0 8~August~2018. Current version (4
      January~2019) 1.0.3.
    \item It is free and open software (GitHub
      \colorlink{https://github.com/JuliaLang/julia}).
    \item Created by Jeff Bezanson, Alan Edelman, Stefan Karpinski,
      and~Viral B.~Shah, most of them work at~MIT at~some stages~of
      their life.
    \item They aim was to~created modern language suited to~numerical
      computing and~scientific use.
    \item Today one~of main center~of development~of the~language
      is~Julia Lab at~MIT, with Alan Edelman as it current head.
    \item They co-founded Julia Computing \emph{to develop products
        that make Julia easy to~use, easy to~deploy and easy
        to~scale}.
    \end{itemize}
  \end{block}

\end{frame}
% ##########





% ##########
\begin{frame}
  \frametitle{Why not learn Julia?}

  \begin{block}{Reasons not to learn}
    \begin{itemize}
    \item[--] You are happy with tools you have now.
    \item[--] You don't find libraries you need.
    \item[--] You don't have time/want learn new programming language.
    \item[--] You don't want to learn new style~of programming.
    \item[--] Since release of \textbf{Julia v1.0}, which clean
      language a~lot, ecosystem~of external packages still need some
      adjustments.
    \item[--] There is~not enough good learning materials
      for~version~1.x.
    \item[--] Still not enough user base.
    \item[--] Ecosystem (packages, IDEs, debugger,\ld) is not as
      mature as~for~other languages/enivroments.
    \item[--] Low ``stackoverflow effect'': how likely answer for your
      question is~under the top link in~Google results.
    \end{itemize}
  \end{block}
  
\end{frame}
% ##########



% ##########
\begin{frame}
  \frametitle{Why and why not learning Julia?}

  \begin{block}{More reasons to~not learn}
    \begin{itemize}
    \item[--] Since Julia is more general purpose language than
      MATLAB, Mathematica,~R, etc., makes finding some things a~little
      bit harder.
    \item[--] No serious project/company use it. Please wait few
      minutes.
    \item[--] Non-trivial performance engineering. You must learn how
      to do that.
    \end{itemize}
  \end{block}

  \begin{block}{Reasons to learn}
    \begin{itemize}
    \item[--] Allows High productivity~of writing code (compilers can
      make more for~us, than 30 years ago).
    \item[--] Allows good readability~of written code (Python
      or~MATLAB like).
    \item[--] Most languages we used are really old created in~1990s,
      1980s, 1970s, 1960s and~1950s. They were designed for very, very
      different computers, environments, to solve different problems
      and~at~different stage~of computer science development.
    \end{itemize}
  \end{block}
  
\end{frame}
% ##########



% % ##########
% \begin{frame}
%   \frametitle{Why learn Julia?}


% \end{frame}
% % ##########



% ##########
\begin{frame}
  \frametitle{Why learn Julia?}

  \begin{block}{Reasons to learn}
    \begin{itemize}
    \item[--] If there is a~mistake in~MATLAB, you must live with that
      and walk around it. In Julia you can correct it yourself.
    \item[--] You can write code closely to~Python way with~REPL
      (shell like thing) and~``scripting'' or~in~browser (like IPython
      Notebook).
    \item[--] After you write your code, in most cases you can
      optimize it~to, achieve 35\%--~105\% speed~of FORTRAN or~C
      (hopefully in~future this gap will became narrower).
    \item[--] Seamless incorporation~of FORTRAN, C, C++, Python,~R
      and~Java.
    \item[--] \emph{Is~Julia the~next big programming language? MIT
        thinks so, as~version~1.0 lands}, see~full from~TechRepublic
      article
      \colorhref{https://www.techrepublic.com/article/is-julia-the-next-big-programming-language-mit-thinks-so-as-version-1-0-lands/}{here}. \\
      Updating \emph{Pan Tadeusz}: What American invent, Pole will
      like.
    \item[--] In Poland we often talking about modernization, so~this
      is good thing to watch what happened at~world level.
    \item[--] New style of programming == new opportunities. You~can
      to some extant write domain specific language.
    \item[--] Community is~very welcoming and~helpful.
    \item[--] Few more.
    \end{itemize}
  \end{block}

\end{frame}
% ##########



% ##########
\begin{frame}
  \frametitle{Who use Julia?}

  \begin{block}{\textbf{Celeste.jl}}
    Project written entirely in~Julia at~National Energy Research
    Scientific Computing Center (NERSC) at Lawrence Berkeley National
    Laboratory (Berkeley Lab). Aim: analyzing data from 35\% of
    visible sky gathered by~Sloan Digital Sky Survey.
    \begin{itemize}
    \item[--] Peak performance of 1.54 petaflops ($10^{ 15 }$ flops)
      using 1.3 million threads on 9,300 Knights Landing nodes~of the
      Cori supercomputer at~NERSC. They say that only assembler, C,
      C++ and FORTRAN achieved previously over 1~petaflops
      performance.
    \item[--] Loaded approximately 178 terabytes of image data and
      give parameters estimates for 188 millions stars and galaxies
      in~14.6 minutes.
    \end{itemize}
    It is worth noting that this computation was performed before
    2018, so~it used 0.x version of the language (probably 0.4
    an~0.5), which has unstable syntax.
  \end{block}

\end{frame}
% ##########




% ##########
\begin{frame}
  \frametitle{Who use Julia?}

  \begin{block}{Big players that use or~give money to support it}
    NASA, Intel, IBM, Google, Microsoft, Amazon, Apple, Alibaba.com,
    Ford, Facebook, Oracle. And~few more.
  \end{block}
  
  \begin{block}{List of institutions, to give some scope}
    MIT Robot Locomotion Group, USA Federal Aviation Administration,
    Federal Reserve Bank of New York, The Brazilian National Institute
    for~Space Research, AOT Energy (energy trading), PSR (global
    electricity and~natural gas consulting, analytic and~technology
    firm), University of Auckland Electric Power Optimization Centre
    (The Milk Output Optimizer), Voxel8 (3D printing), University of
    Copenhagen, Peking University and Imperial College London
    collaboration on \emph{An~Anthropocene Map~of Genetic Diversity}
    (Science, vol 343, issue 6307, pp. 1532-1535), Path BioAnalytics
    (medicine).
  \end{block}

  \begin{block}{}
    See for more cases and details here
    \colorlink{https://juliacomputing.com/case-studies/} and here
    \colorlink{https://www.forbes.com/sites/suparnadutt/2017/09/20/this-startup-created-a-new-programming-language-now-used-by-the-worlds-biggest-companies/\#7f322c907de2}.
  \end{block}

\end{frame}
% ##########





% ######################################
\section[]{Lies, big lies and~benchmarks}
% ######################################




% ##########
\begin{frame}
  \frametitle{There are lies, big lies and benchmarks}

  \begin{block}{Compression of different languages}
    \begin{figure}
      \centering

      \includegraphics[scale=0.22]{Julia_micro_benchmarks.png}
      \caption{From page \emph{Julia Micro-Benchmarks},
        \colorhref{https://julialang.org/benchmarks/}{https://julialang.org/benchmarks/}.}
    \end{figure}

    \textbf{Warning!} Python code use numpy libraries that is~written
    52.8\% in C (state of GitHub repository at~4 January~2019).
  \end{block}
  
\end{frame}
% ##########



% ##########
\begin{frame}
  \frametitle{Solving numerical PDEs, how good Julia can be?}

  \begin{block}{Example: Kuramoto-Sivashinsky equation in~$1 + 1$ dimensions}
    Kuramoto-Sivashinsky equation is nonlinear (more precisely:
    semilinear) partial differential equation, which in $1 + 1$
    dimension takes form
    \begin{equation}
      \label{eq:1}
      \pd{ }{ u( t, x ) }{ t } + \pd{ 4 }{ u( t, x ) }{ x }
      + \pd{ 2 }{ u( t, x ) }{ x } + u( t, x ) \pd{ }{ u( t, x ) }{ x }
      = 0.
    \end{equation}
    Full $1 + 3$ dimensional version~of this equation was proposed
    to~\emph{model the~diffusion instabilities in~a~laminar flame
      front}.
  \end{block}

  \begin{block}{There are lies, big lies and benchmarks}
    This benchmarks were crated by~John F.~Gibson, Dept. Mathematics
    and~Statistics, University~of New Hampshire, main author~of
    \textbf{Channelflow}: \\
    \emph{a~set of high-level software tools and~libraries for
      research in~turbulence in~channel geometries} written in~C++,
    \colorlink{http://channelflow.org/}. Full list~of people that
    contribute to~benchmarks is
    \colorhref{https://github.com/johnfgibson/julia-pde-benchmark/blob/master/1-Kuramoto-Sivashinksy-benchmark.ipynb}{here}.
  \end{block}

\end{frame}
% ##########





% ##########
\begin{frame}
  \frametitle{Solving numerical PDEs, how good Julia can be?}

  \begin{block}{They are outdated}
    They are from middle 2017, so~they using Julia 0.5 (?) and as such
    they are outdated. For this reason I don't present any code, only
    results. John~F.~Gibson said that he will update them to
    Julia~1.x, but~they were not ready for this presentation
    (or~I~don't notice update in~last week).
  \end{block}

  \begin{block}{If you want to see code yourself}
    All code in all languages used for creating this benchmarks
    is~available on~GitHub \colorhref{https://github.com/johnfgibson/julia-pde-benchmark}{johnfgibson/julia-pde-benchmark}. \\
    \textbf{Warning!} Python code use numpy libraries that is~written
    52.8\% in C (state of GitHub repository at~4 January~2019).
  \end{block}

  % \begin{block}{Algorithm}
  %   Description of used numerical algorithm can be~founded
  %   \colorhref{https://github.com/johnfgibson/julia-pde-benchmark/blob/master/1-Kuramoto-Sivashinksy-benchmark.ipynb}{here}.
  % \end{block}

\end{frame}
% ##########


% ##########
\begin{frame}
  \frametitle{Solving numerical PDEs, how good Julia can be?}

  \begin{block}{Algorithm}
    \emph{The KS-CNAB2 benchmark algorithm is~a~simple numerical
      integration scheme for~the~KS [Kuramoto-Sivashinsky] equation
      that uses Fourier expansion in~space [All codes call this same
      Fourier Transform external library.], collocation calculation~of
      the~nonlinear term $u( t, x ) u_{ x }( t, x )$,
      and~finite-differencing in~time, specifically 2nd-order
      Crank-Nicolson Adams-Bashforth
      (CNAB2) timestepping.} \\
  \end{block}

  \begin{block}{}
    More details on~algorithms can be found
    \colorhref{https://github.com/johnfgibson/julia-pde-benchmark/blob/master/1-Kuramoto-Sivashinksy-benchmark.ipynb}{here}.
  \end{block}

\end{frame}
% ##########





% ##########
\begin{frame}
  \frametitle{Solving numerical PDEs, how good Julia can be?}

  \begin{block}{Result}
    \begin{figure}
      \centering

      \includegraphics[scale=0.22]{KS_result.png}
      \caption{Kuramoto-Sivashinsky heat evolution in~1~dimension.}
    \end{figure}
  \end{block}

\end{frame}
% ##########





% ##########
\begin{frame}
  \frametitle{Solving numerical PDEs, how good Julia can be?}

  \begin{block}{Chart with linear scale}
    \begin{figure}
      \centering

      \includegraphics[scale=0.22]{JFG_linear_scale.png}
      \caption{Results for $N_{ x }$ points on $x$ axis, CPU time in
        seconds.}
    \end{figure}
  \end{block}

\end{frame}
% ##########





% ##########
\begin{frame}
  \frametitle{Solving numerical PDEs, how good Julia can be?}

  \begin{block}{Chart with logarithmic scale}
    \begin{figure}
      \centering

      \includegraphics[scale=0.22]{JFG_logarithm_scale.png}
      \caption{Results for $N_{ x }$ points on $x$ axis, CPU time in
        seconds.}
    \end{figure}
  \end{block}

\end{frame}
% ##########





% ##########
\begin{frame}
  \frametitle{Solving numerical PDEs, how good Julia can be?}

  \begin{block}{Time for maximal $N_{ x } = 2^{ 17 } = 131072$}
    \begin{table}
      \centering
      
      \begin{tabular}{|l|r|r|}
        \hline
        Language & CPU time [s] & Ratio to~C \\
        \hline
        Fortran & 13.0 & 0.90 \\
        \hline
        Julia unrolled & 13.6 & 0.93 \\
        \hline
        C++ & 14.4 & 0.99 \\
        \hline
        C & 14.6 & 1.00 \\
        \hline
        Julia in-place & 15.5 & 1.06 \\
        \hline
        Julia naive & 23.1 & 1.58 \\
        \hline
        MATLAB & 26.8 & 1.83 \\
        \hline
        Python & 35.7 & 2.45 \\
        \hline
      \end{tabular}
      \caption{From the~fastest to~the~slowest code.}
    \end{table}
  \end{block}

\end{frame}
% ##########





% ##########
\begin{frame}
  \frametitle{Solving numerical PDEs, how good Julia can be?}

  \begin{block}{Time vs~lines~of code}
    \begin{figure}
      \centering

      \includegraphics[scale=0.22]{JFG_time_vs_code.png}
      \caption{Compression~of time and~lines~of code.}
    \end{figure}
  \end{block}

\end{frame}
% ##########





% ##########
\begin{frame}
  \frametitle{Solving numerical PDEs, how good Julia can be?}

  \begin{block}{Speed/lines~of code}
    \begin{figure}
      \centering

      \includegraphics[scale=0.22]{JFG_speed_over_code.png}
      \caption{Compression~of ratio~of speed to the lines~of code.}
    \end{figure}
  \end{block}

\end{frame}
% ##########




% ##########
\begin{frame}
  \frametitle{Closing remarks}

  \begin{block}{Quite good summary}
    \emph{Julia is still not great when you want import some libraries
      and use them. It~is great when you must write new library.}
    Chris Rackauckas on~Julia Discourse.
  \end{block}

  \begin{block}{My reflections}
    \begin{itemize}
    \item Julia proved that it~can be~used to~make serious scientific
      work in~various fields, including numerical computations
      relevant to~field theory.
    \item Pedagogical perspective. Learning and~using it~is comparable
      to~Python (in my~opinion), which is~starting language for
      computer scientists on~MIT and~at~the same time is~more deep
      as~computer language and~have more computational power.
    \item Have potential to~make money outside academic realm.
    \end{itemize}
  \end{block}

\end{frame}
% ##########





% ##########
\begin{frame}
  \frametitle{Closing remarks}

  \begin{block}{If you use it in your research}
    They ask you for citing paper \textbf{Julia: A~Fresh Approach to
      Numerical Computing}, Jeff Bezanson, Alan Edelman, Stefan
    Karpinski and~Viral B.~Shah~(2017) SIAM Review, 59:~65--98.
    doi:~10.1137/141000671,
    url:~http://julialang.org/publications/julia-fresh-approach-BEKS.pdf,
    \colorlink{https://julialang.org/publications/julia-fresh-approach-BEKS.pdf}.
    And add you paper to the following list
    https://julialang.org/publications/,
    \colorlink{https://julialang.org/publications/}.
  \end{block}

  \begin{block}{I want to thank these people for help and discusions}
    \begin{itemize}
    \item Tamas Papp,
    \item John F. Gibson,
    \item Yakir Luc Gagnon,
    \item Antoine Levitt,
    \item Krzysztof Musiał.
    \end{itemize}
    With most of them I converse on Julia Discourse,
    \colorlink{https://discourse.julialang.org/}.
  \end{block}

\end{frame}
% ##########





% ##########
\begin{frame}

  \begin{center}
    \LARGE Thank you.
  \end{center}

\end{frame}
% ##########







% % ######################################
% \section[]{Regularyzacja wymiarowa w~przestrzeni położeń}
% % ######################################





% ##########
\begin{frame}
  \frametitle{Few interesting things for which there is no time here}

  \begin{block}{Packages and projects}
    \begin{itemize}
    \item DynamicalSystems.jl, GitHub
      \colorhref{https://github.com/JuliaDynamics/DynamicalSystems.jl}{JuliaDynamics/DynamicalSystems.jl}.
      Winner~of one~of three main prizes of The~Dynamical System Web
      in~2018.
    \item QuantumOptics.jl. Numerical framework for numerical solving
      open quantum systems, more on this page
      \colorhref{https://qojulia.org/}{https://qojulia.org/}.
    \item Machine Learning on Google's Cloud TPUs (tensor processing
      units). \\
      \emph{Targeting TPUs using our compiler, we are able to evaluate
        the~VGG19 forward pass on~a~batch~of 100 images in~0.23s which
        compares favorably to~the~52.4s required for~the~original
        model on~the~CPU. Our implementation is~less than 1000
        lines~of Julia, \\
        with no~TPU specific changes made to~the~core
        Julia compiler \\
        or~any other Julia packages.} Whole article by Keno Fischer
      and Elliot Saba
      \colorhref{https://arxiv.org/abs/1810.09868}{here}.
    \end{itemize}
  \end{block}

\end{frame}
% ##########





% ##########
\begin{frame}
  \frametitle{Few interesting things for which there is no time here}

  \begin{block}{Things that are important or~promising}
    \begin{itemize}
    \item Type system,
      \colorlink{https://www.youtube.com/watch?v=Z2LtJUe1q8c}.
    \item Metaprogramming and~how use it to make code efficient,
      \colorlink{https://www.youtube.com/watch?v=SeqAQHKLNj4}.
    \item Multiple dispatch as rare way of Object Oriented Programming
      (I~don't want holy war over this statement),
      \colorlink{https://www.youtube.com/watch?v=gZJFHrYopxw}.
    \item Using GPU,
      \colorlink{https://www.youtube.com/watch?v=6ntJ_al4oXA}.
    \item Julia package manager,
      \colorlink{https://www.youtube.com/watch?v=HgFmiT5p0zU}.
    \item Symbolic computing like in~Wolfram Mathematica,
      \colorlink{https://www.youtube.com/watch?v=M742_73edLA}.
    \item Monte Carlo, very immature,
      \colorlink{https://www.youtube.com/watch?v=BmVd7pw6Trc}.
    \item Writing fast code,
      \colorlink{https://www.youtube.com/watch?v=szE4txAD8mk}
    \item Parallel computing,
      \colorlink{https://www.youtube.com/watch?v=euZkvgx0fG8}.
    \item Regulars expressions,
      \colorlink{https://docs.julialang.org/en/v1/manual/strings/}.
    \item Tensor compilers,
      \colorlink{https://www.youtube.com/watch?v=Rp7sTl9oPNI}.
    \item How Julia work inside,
      \colorlink{https://www.youtube.com/watch?v=7KGZ_9D_DbI}.
    \end{itemize}
  \end{block}
  
\end{frame}
% ##########





% % ##########
% \begin{frame}
%   \frametitle{Lies, big lies and benchmarks from 2017}

%   \begin{block}{They are outdated now, but there aren't any newer}
%     \begin{figure}
%       \centering

%       \includegraphics[scale=0.7]{benchmarks_QO.pdf}
%       \caption{QuTiP --~Quantum Toolbox in Python.}
%     \end{figure}
%     More benchmarks on page ?????
%   \end{block}

%   \begin{block}{Article to cite}
  
%   \end{block}

% \end{frame}
% % ##########





% ######################################
\section[]{Bibliography and~resources}
% ######################################





% ##########
\begin{frame}
  \frametitle{Bibliography}

  \begin{block}{Relevant articles}
    \begin{itemize}
    \item Jeff Bezanson, Alan Edelman, Stefan Karpinski and~Viral
      B.~Shah, \emph{Julia: A~Fresh Approach to Numerical Computing},
      (2017) SIAM Review, 59:~65--98. doi:~10.1137/141000671,
      url:~http://julialang.org/publications/julia-fresh-approach-BEKS.pdf,
      \colorlink{https://julialang.org/publications/julia-fresh-approach-BEKS.pdf}.
    \item Chistopher Rackauckas and Qing~Nie,
      \emph{DifferentialEquations.jl --~A~Performant and~Feature-Rich
        Ecosystem for Solving Differential Equations in~Julia}.
      Journal~of Open Research Software (2017). 5(1), p.15. doi:
      http://doi.org/10.5334/jors.151,
      \colorlink{https://openresearchsoftware.metajnl.com/articles/10.5334/jors.151/}.
    \item Keno Fischer, Elliot Saba, \emph{Automatic Full
        Compilation~of Julia Programs and~ML Models to~Cloud TPUs},
      \arXiv{https://arxiv.org/abs/1810.09868}{1810.09868}.
    \item Nick Heath, \emph{Is~Julia the~next big programming
        language? \\
        MIT thinks so, as~version~1.0~lands}. \\
      TechRepublic, August~29, 2018,
      \colorlink{https://www.techrepublic.com/article/is-julia-the-next-big-programming-language-mit-thinks-so-as-version-1-0-lands/}.
    \end{itemize}
  \end{block}
  
\end{frame}
% ##########





% ##########
\begin{frame}
  \frametitle{Netography}

  \begin{block}{}
    \begin{itemize}
    \item JuliaLang/julia -- Julia GitHub repository,
      \colorlink{https://github.com/JuliaLang/julia}.
    \item numpy/numpy -- numpy GitHub repository,
      \colorlink{https://github.com/numpy/numpy}.
    \item John F.~Gibson, \emph{julia-pde-benchmark}, \\
      GitHub
      \colorhref{https://github.com/johnfgibson/julia-pde-benchmark}{johnfgibson/julia-pde-benchmark}.
    \item John F.~Gibson, \emph{Why~Julia}, GitHub
      \colorhref{https://github.com/johnfgibson/whyjulia/blob/master/1-whyjulia.ipynb}{johnfgibson/whyjulia}.
    \end{itemize}
  \end{block}

\end{frame}
% ##########





% ##########
\begin{frame}
  \frametitle{Learning materials}

  \begin{block}{Basics materials}
    Most~of them are more or~less outdated, since they were created
    before release~of Julia 1.x.
    \begin{itemize}
    \item JuliaBoxTutorials, version 1.x, GitHub
      \colorhref{https://github.com/JuliaComputing/JuliaBoxTutorials}{JuliaComputing/JuliaBoxTutorials}.
      Good starting point.
    \item Julia 1.x Documentation. Always~up to~date, really good
      written in~comparison to~others manuals,
      \colorhref{https://docs.julialang.org/en/v1/}{https://docs.julialang.org/en/v1/}.
    \item David P.~Sanders, \emph{Introduction to~Julia for~scientific
        Computing (Workshop)}, 2015. Outdated, but very good
      introduction to~language,
      \YouTube{https://www.youtube.com/watch?v=gQ1y5NUD_RI}.
    \item The Julia Language channel
      on~\colorhref{https://www.youtube.com/user/JuliaLanguage}{YouTube}.
      Contains dozens videos from JuliaCons and hold regulars
      \emph{Intro to~Julia}, keeping it up to date. You~can find next
      \emph{Intro to~Julia} and other introductions
      on~\colorhref{https://www.facebook.com/Learn-Julia-529467964069525/?eid=ARDnSanRh6e3_6NQanjfziNeMYt3-hRUqje-5xOvRidGM5bmm6ZGCZ49fy5CZ1AYX9T503OEdMUMRlOe}{Facebook}.
    \item Ben Lauwens, Allen Downey, \emph{Think Julia: How to~Think
        Like a~Computer Scientist},
      \colorlink{https://benlauwens.github.io/ThinkJulia.jl/latest/book.html}.
    \end{itemize}
  \end{block}

\end{frame}
% ##########





% ##########
\begin{frame}
  \frametitle{Learning materials}

  \begin{block}{Practical introductions to many different topics}
    JuliaCon is annual conference on~Julia language which start
    at~2014.
    \begin{itemize}
    \item Stefan Karpinski and~Kristoffer Carlsson,
      \emph{Pkg3:~The~new Julia package manager}, JuliaCon 2018,
      \YouTube{https://www.youtube.com/watch?v=HgFmiT5p0zU}.
    \item Arch D. Robison, \emph{Introduction to~Writing High
        Performance Julia (Workshop)}, JuliaCon 2016,
      \YouTube{https://www.youtube.com/watch?v=szE4txAD8mk}.
    \item Andy Ferris, \emph{A~practical introduction
        to~metaprogramming in~Julia}, JuliaCon 2018,
      \YouTube{https://www.youtube.com/watch?v=SeqAQHKLNj4}.
    \item Chris Rackauckas, \emph{Intro to solving differential
        equations in Julia},
      \YouTube{https://www.youtube.com/watch?v=KPEqYtEd-zY}.
    \item DiffEqTutorials.jl. Tutorials~of JuliaDiffEq project, GitHub
      \colorhref{https://github.com/JuliaDiffEq/DiffEqTutorials.jl}{JuliaDiffEq/DiffEqTutorials.jl}.
    \end{itemize}
  \end{block}

\end{frame}
% ##########





% ##########
\begin{frame}
  \frametitle{Learning materials}

  \begin{block}{More theoretical, less practical materials}
    \begin{itemize}
    \item Jeff Bezanson, \emph{Why is~Julia fast?}, 2015,
      \YouTube{https://www.youtube.com/watch?v=cjzcYM9YhwA}.
    \item Jeff Bezanson, \emph{The~State~of the~Type System}, JuliaCon
      2017, \YouTube{https://www.youtube.com/watch?v=Z2LtJUe1q8c}.
    \item Jiahao Chen, \emph{Why language matters: Julia and~multiple
        dispatch}, 2016,
      \YouTube{https://www.youtube.com/watch?v=gZJFHrYopxw}.
    \item Jameson Nash, \emph{AoT or~JIT: How Does Julia Work?}, (AoT
      --~Ahead~of Time compilation, JIT --~Just In~Time compilation),
      JuliaCon 2017,
      \YouTube{https://www.youtube.com/watch?v=7KGZ_9D_DbI}.
    \item John Lapyre, \emph{Symbolic Mathematics in Julia}, JuliaCon
      2018, \YouTube{https://www.youtube.com/watch?v=M742_73edLA}.
    \item Julia Lab at~MIT, \emph{Parallel Computing (Workshop)},
      JuliaCon 2016,
      \YouTube{https://www.youtube.com/watch?v=euZkvgx0fG8}.
    \item Tim Besard, Valentin Churavy and Simon Danisch,
      \emph{GPU~Programming with~Julia}, JuliaCon 2017,
      \YouTube{https://www.youtube.com/watch?v=6ntJ_al4oXA}.
    \end{itemize}
  \end{block}

\end{frame}
% ##########





% ##########
\begin{frame}
  \frametitle{Learning materials}

  \begin{block}{More theoretical, less practical materials}
    \begin{itemize}
    \item Peter Ahrens, \emph{For~Loops~2.0: Index Notation
        And~The~Future~Of Tensor Compilers}, JuliaCon 2018,
      \YouTube{https://www.youtube.com/watch?v=Rp7sTl9oPNI}.
    \item Carsten Bauer, \emph{Julia for Physics: Quantum Monte
        Carlo}, JuliaCon 2018,
      \YouTube{https://www.youtube.com/watch?v=BmVd7pw6Trc}.
    \item George Datseris, \emph{Why Julia is the most suitable
        language for science}, case study~of project JuliaDynamics,
      \colorlink{https://www.youtube.com/watch?v=7y-ahkUsIrY}.
    \item Nick Higham, \emph{Tricks and~Tips in~Numerical Computing},
      JuliaCon 2018,
      \YouTube{https://www.youtube.com/watch?v=Q9OLOqEhc64}.
    \end{itemize}
  \end{block}
  
\end{frame}
% ##########





% ##########
\begin{frame}
  \frametitle{Mentioned projects and articles}

  \begin{block}{Written Julia if not mentioned otherwise}
    Still many~of them don't work in Julia 1.x.
    \begin{itemize}
    \item[--] Channelflow (written in~C++),
      \colorhref{http://channelflow.org/}{http://channelflow.org/}.
    \item[--] DynamicalSystems.jl, GitHub
      \colorhref{https://github.com/JuliaDynamics/DynamicalSystems.jl}{JuliaDynamics/DynamicalSystems.jl}.
    \item[--] QuantumOptics.jl. Numerical framework for numerical
      solving open quantum systems, more on this page
      \colorhref{https://qojulia.org/}{https://qojulia.org/}.
    \item[--] REPL (shell) for C++ (broken in Julia 1.x), GitHub
      \colorhref{https://github.com/NHDaly/PaddleBattleJL}{Keno/Cxx.jl}.
    \item[--] Game and educational tool \emph{Paddle Battle}, GitHub
      \colorhref{https://github.com/NHDaly/PaddleBattleJL}{NHDaly/PaddleBattleJL}.
    \end{itemize}
  \end{block}

\end{frame}
% ##########





% ######################################
\section[]{Additional information and topics}
% ######################################





% ##########
\begin{frame}

  \begin{center}
    \LARGE Additional information and topics
  \end{center}

\end{frame}
% ##########





% ##########
\begin{frame}
  \frametitle{Current state~of Language}

  \begin{block}{Some statistics}
    \begin{itemize}
    \item[--] Google Scholar: 607 papers about Julia or using it in
      research (state at~20 November 2018). See also topic
      \emph{Research} at~Julia Language page
      https://julialang.org/publications/,
      \colorlink{https://julialang.org/publications/}.
    \item[--] Used in some way in over 1,000 universities.
    \item[--] Ecosystem of over 1,900 packages for wide are of topics.
    \item[--] Over 2~millions downloads.
    \item[--] Over 41,000 GitHub stars for language and packages.
    \item[--] 101\% annual growth (based on~downloads).
    \end{itemize}
    Outside first positions rest all others are taken from Julia
    Computing and~dated from August~2018.
  \end{block}

\end{frame}
% ##########





% ##########
\begin{frame}
  \frametitle{Current state~of language}

  \begin{block}{Julia source code}
    \begin{table}
      \centering
      
      \begin{tabular}{|l|r|r|}
        \hline
        Language & Percent of code \\
        \hline
        Julia & 68.4\% \\
        \hline
        C & 16.6\% \\
        \hline
        C++ & 10.0\% \\
        \hline
        Scheme & 3.4\% \\
        \hline
        Makefile & 0.6\% \\
        \hline
        Shell & 0.3\% \\
        \hline
        Other & 0.7\% \\
        \hline
      \end{tabular}
      \caption{Numbers from GitHub JuliaLang/julia, 4~January~2019.}
    \end{table}
  \end{block}

  \begin{block}{}
    Scheme --~Lisp dialect from
    MIT.% , LLVM --~Low Level Virtual Machine.
  \end{block}
  
\end{frame}
% ##########





% ##########
\begin{frame}
  \frametitle{Cxx.jl (probably still broken in~1.x)}

  \begin{block}{RELP for C++}
    \begin{figure}
      \centering

      \includegraphics[scale=0.29]{Cxx-jl.png}
      \caption{Compression~of ratio~of speed to the lines~of code.}
    \end{figure}
  \end{block}

\end{frame}
% ##########





% ##########
\begin{frame}
  \frametitle{Application in Julia}

  % \begin{block}{}
  \begin{figure}
    \centering

    \includegraphics[scale=0.17]{PaddleBattle.png}
    % \caption{}
  \end{figure}
  % \end{block}

  \begin{block}{Corollary}
    One man write game in pure Julia (with some shell scripts,
    to~automatize building~up binaries) and put it on~Mac App Store,
    to~show if~you want, you~can do it quite easily. Code of game is
    available on
    \colorhref{https://github.com/NHDaly/PaddleBattleJL}{here}.
  \end{block}

\end{frame}
% ##########





% ##########
\begin{frame}
  \frametitle{Mentioned projects and articles}

  \begin{block}{Written Julia if not mentioned otherwise}
    Still many~of them don't work in Julia 1.x.
    \begin{itemize}
    \item[--] Channelflow (written in~C++),
      \colorhref{http://channelflow.org/}{http://channelflow.org/}.
    \item[--] DynamicalSystems.jl, GitHub
      \colorhref{https://github.com/JuliaDynamics/DynamicalSystems.jl}{JuliaDynamics/DynamicalSystems.jl}.
    \item[--] QuantumOptics.jl. Numerical framework for numerical
      solving open quantum systems, more on this page
      \colorhref{https://qojulia.org/}{https://qojulia.org/}.
    \item[--] REPL (shell) for C++ (broken in Julia 1.x), GitHub
      \colorhref{https://github.com/NHDaly/PaddleBattleJL}{Keno/Cxx.jl}.
    \item[--] Game and educational tool \emph{Paddle Battle}, GitHub
      \colorhref{https://github.com/NHDaly/PaddleBattleJL}{NHDaly/PaddleBattleJL}.
    \end{itemize}
  \end{block}

\end{frame}
% ##########






% ####################################################################
% ####################################################################
% Bibliografia
\bibliographystyle{alpha} \bibliography{Bibliography}{}


% ############################

% Koniec dokumentu
\end{document}
